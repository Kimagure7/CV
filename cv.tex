\documentclass[a4paper, 11pt]{moderncv}

\moderncvstyle{classic}
\moderncvcolor{blue}
\nopagenumbers{}

\usepackage{ctex}

% adjust the page margins
\usepackage[scale=0.85]{geometry}
\recomputelengths

\firstname{赵}
\familyname{子毅}
\phone{(+86)~136~6243~7820}
\email{re2477036@mail.ustc.edu.cn}
\extrainfo{\githubsocialsymbol \href{https://github.com/Kimagure7}{Kimagure7}}

\begin{document}
\makecvtitle

\section{教育背景}
\cventry{2020~--~2024}{中国科学技术大学}{计算机科学与技术专业}{}{}{
    \begin{itemize}
        \item 总绩点: 3.73/4.30, 专业排名: 50/262
        \item 相关课程:
              \begin{itemize}
                  \item 编译原理与技术 (96)
                  \item 算法基础 (94)
                  \item 操作系统原理与设计 (90)
                  \item 计算机组成原理 (91)
                  \item 数据结构 (92)
              \end{itemize}
    \end{itemize}
}

\section{项目经历}
\cventry{2023.01}{CminusF 编译器}{编译原理与技术课程实验}{}{}{
    \begin{itemize}
        \item 实现了其前端 parser, IR 生成和部分中端代码优化.
        \item 实现了基于数据流分析的 GVN 分析, 用于常量传播和删除冗余代码.
        \item 使用 LLVM 后端生成机器码.
    \end{itemize}
}

\cventry{2022.05}{RISC-V CPU 核}{计算机组成原理课程大作业}{}{}{
    \begin{itemize}
        \item 五级流水线的 32-bit RISC-V CPU 核心, 带有 2-bit 动态分支预测器.
        \item 实现了大部分 RV32I 指令, 包括算术, 逻辑, 访存和控制指令.
        \item 能正确在 FPGA 开发板上运行.
    \end{itemize}
}

\section{获奖荣誉}

\cvitem{2023.01}{江淮蔚来汽车奖学金}
\cvitem{2022.01}{陈林义奖学金}
\cvitem{2020.12}{优秀学生奖学金银奖}

\vspace{1em}

\cvitem{2022.4}{中国科学技术大学程序设计竞赛 (USTCPC) Div 2 组别三等奖}

\section{专业技能}
\subsection{编程语言}
\cvitem{熟练}{C/C++, Python, Verilog, Markdown, \LaTeX}
\cvitem{了解}{Rust, Lua, Shell Script}

\subsection{英语}
\cvitem{}{四级 620, 六级 546}

\end{document}