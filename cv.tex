\documentclass[a4paper, 11pt]{moderncv}

\moderncvstyle{classic}
\moderncvcolor{blue}
\nopagenumbers{}

\usepackage{ctex}

% adjust the page margins
\usepackage[scale=0.85]{geometry}
\recomputelengths

\firstname{赵}
\familyname{子毅}
\phone{(+86)~136~6243~7820}
\email{re2477036@mail.ustc.edu.cn}
\extrainfo{\githubsocialsymbol \href{https://github.com/Kimagure7}{Kimagure7}}

\begin{document}
\makecvtitle

\section{教育背景}
\cventry{2020~--~2024}{中国科学技术大学}{计算机科学与技术专业}{}{}{
    \begin{itemize}
        \item 总绩点: 3.73/4.30, 专业排名: 39/186
        \item 相关课程:
              \begin{itemize}
                  \item 编译原理与技术 (95)
                  \item 线性代数B1 (96)
                  \item 操作系统原理与设计H (96)
                  \item 计算机网络 (95)
                  \item 数据结构 (90)
              \end{itemize}
    \end{itemize}
}

\section{项目经历}
\cventry{2022.05}{Graph Network Disk with Prometheus}{操作系统原理与设计H课程大作业(合作)}{}{}{
    \begin{itemize}
        \item 一个带有图视图, 列表视图的分布式网盘. 可以自动对上传文件打标以分类, 后续可在图视图中直观看到文件之间的联系.
        \item 本人负责图数据库和图视图部分.
    \end{itemize}
}
\cventry{2023.04 ~-~ 2023.06}{初探大模型训练}{中科大-微软 创新实践项目}{}{}{
    \begin{itemize}
        \item 学习Swin Transformer的基本原理
        \item 探索Swin Transformer的优化方式
    \end{itemize}
}

\section{获奖荣誉}

\cvitem{2022.10}{优秀学生奖学金银奖}
\cvitem{2021.10}{优秀学生奖学金银奖}

\section{专业技能}
\subsection{编程语言}
\cvitem{熟练}{C/C++, Python}

\subsection{英语}
\cvitem{}{四级 556, 六级 576}

\end{document}